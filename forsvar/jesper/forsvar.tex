\documentclass[t]{beamer}

\usepackage[utf8]{inputenc}
\usepackage[T1]{fontenc}
\usepackage[danish]{babel}

\usetheme{Warsaw}
\usenavigationsymbolstemplate{}

\title{Fladuino}
\subtitle{Funktionel reaktiv programmering på indlejrede enheder}

\author{Martin Dybdal, \\
Troels Henriksen og \\
Jesper Reenberg
}

\begin{document}


\frame{\titlepage}

\section[Outline]{}
\frame{\tableofcontents}


\section{Platforms}

\subsection{Indledning}

\begin{frame}
  \frametitle{Hvilke}
  
  ... problemer løses?

  \begin{itemize}
  \item Kan kun bruge enheder ("`devices"') der passer til den givne platform
    \begin{itemize}
    \item Capabilities: BlueTooth, 3pi motors, ...
    \end{itemize}
  \item Kan kun forbinde enheder med kompatible hardware pins.
    \begin{itemize}
    \item Pin funktionalitet: PWM, interrupt, analog, ...
    \end{itemize}
  \item Giver "`type"' sikkerhed for brug af enheder på en platform
  \end{itemize}  

  En platform vælges i make filen og gives som argument til \textit{fladuinogen}

\end{frame}


\begin{frame}
  \frametitle{Opsummering}

  Hvad indeholder en platforms specifikation?
  \begin{description}
  \item [Pins] Liste af analog og digitale pins
  \item [Pinmap] mapping funktion mellem Arduino pins og Logical pins
  \item [Capabilities] Liste af platformens funktionaliteter 
  \item [Base setup] Initialisering der skal ske i Arduinos \texttt{setup()}
    funktion, fx \texttt{addCInitStm} til Arduino BT
  \item [Timer id] Nummer på \texttt{clock}-timer
  \end{description}
\end{frame}

\subsection{Implementering}

\begin{frame}
  \frametitle{Håndtering}
  
  \begin{itemize}
  \item Alle funktioner der bruger enheder kalder funktionen \texttt{addDevice},
    hvor enheden tilføjes til en enhedsliste hvis de ikke allerede er tilføjet.
    
  \item Dette kunne være \texttt{toggle}, \texttt{turnOn} og \texttt{turnOff}
    som tager \texttt{DigitalOutputPin} enheden som argument. \texttt{Diode}
    enheden er et eksempel på en \texttt{DigitalOutputPin}.
    
  \item Ved generation-time kaldes bl.a. funktionerne
    \begin{itemize}
    \item finalizeDevices
    \item finalizeConfig
    \end{itemize}
  \end{itemize}
\end{frame}

\begin{frame}
  \frametitle{\texttt{finalizeDevices}}

  \begin{itemize}
  \item Tager den givne platform som argument

  \item Generer en konfiguration indeholdende 
      \begin{itemize}
      \item \alt<2> {Den givne platform}{$\ldots$}
      \item<only@2> Tom liste af pin konfigurationer 
      \item<only@2> Setup liste indeholdende platformens base setup.
      \end{itemize}
  \item For hver enhed i enhedslisten tilføj pin konfigurationer hvis
    \begin{itemize}
    \item \alt<3> {for alle enhedens pin konfigurationer:}{$\ldots$}
      \begin{itemize}[<only@3>]
      \item En anden pin konfiguration i listen ikke allerede bruger denne
        pin.

      \item Den givne pin eksisterer i platformens liste af pins

      \item Den givne pins krav er understøttet af platformen (fx PWM,
        interrupt, analog).

      \item Eller \texttt{CapabilityRequired} er understøttet af platformen.
      \end{itemize}
      
    \item<only@3> Hvis ikke dette er opfyldt stoppes genereringen med en
      fejlmeddelelse.

    \item<only@3> Tilføj enhedens setup kode til konfigurationens listen af
      setup kode
    \end{itemize}
  \end{itemize}

\end{frame}

\begin{frame}
  \frametitle{\texttt{finalizeConfig}}
  
  \begin{itemize}
  \item Bruger den før genererede konfiguration

  \item Udvider konfigurationens setup liste ud fra pin konfigurationer

  \item Digital pins bliver initialiseret som digital \texttt{INPUT},
    \texttt{OUTPUT} eller analog \texttt{OUTPUT} (PWM)
  \end{itemize}

\end{frame}

\subsection{Evaluering}

\begin{frame}
  \frametitle{Brugbarhed}
  
  \begin{itemize}
  \item Giver "`type"' tjek af enheder og platforme

  \item Flere enheder "`ens"'/ligende enheder kan ikke bruge samme pins.
  \end{itemize}


  !!!!!! FUTURE WORK!!!!!!

\end{frame}

\section{Posting Wires}

\subsection{Indledning}

\begin{frame}
  \frametitle{Hvorfor?}

  \begin{itemize}
  \item Ideen stammer oprindeligt fra den første Flask artikkel implementeret i
    OCaml

  \item Løser problemet med blocking og busy-waiting.
  \end{itemize}

!!!!!!!!!!!!!!!!!  hvilke? Tag problemstilling fra oprindelig flask artikkel.

\end{frame}


  
\begin{frame}
  \frametitle{Opsummering}

  Foreslåede løsninger til problemstillingen:

  \begin{itemize}
  \item Akkumuleret gentagelse.
  \item Kø
  \item Ignorer
  \end{itemize}

\end{frame}

\subsection{Implementering}

\begin{frame}
  \frametitle{\texttt{postThenWait}}

  \begin{itemize}
  \item Grund byggestenen i posting wires.
  \item Tager to argumenter
    \begin{itemize}
    \item Funktion som har ansvaret at starte dataindsamling. Kan også gøre
      ingen ting.
    \item Event som skal ventes på, før dataflow fortsættes.
    \end{itemize}
  \end{itemize}

  Når dataflow grafen når til \texttt{postThenWait}

  \begin{itemize}
  \item Indikations flag sættes lig 1
  \item Dataflow stoppes
  \end{itemize}

  Når event hænder

  \begin{itemize}
  \item Hvis indikations flag er lig 1
    \begin{itemize}
    \item Sæt indikatios flag lig 0
    \item Dataflow fortsættes.
    \end{itemize}
  \end{itemize}


\end{frame}

\begin{frame}[fragile]
  \frametitle{\texttt{delayUntilEvent}}

  Eksempel på abstraktion af \texttt{postThenWait} defineret som

\begin{verbatim}
  delayUntilEvent = postThenWait [$exp|\_ -> ()|]
\end{verbatim}

  \begin{example}
\begin{verbatim}
onEvent (PushButtonPressEvent $ PushButton 3) 
            >>> delayUntilEvent (PushButtonPressEvent 
                                 $ PushButton 2)
            >>> toggle (diode 8 True)
\end{verbatim}
  \end{example}

  Intet sker ved tryk på knap(2), før end knap(3) er trykket og derefter
  knap(2), hvilket toggler dioden.

\end{frame}


\subsection{Evaluering}

\begin{frame}
  \frametitle{Brugbarhed}
  
  \begin{itemize}
  \item Elegant hvis der skal laves passiv venten.
    \begin{itemize}
    \item Fx: Ubåd sender sonar og skal vente på ekko.
    \end{itemize}
  \end{itemize}
\end{frame}


\end{document}


%%% Local Variables: 
%%% mode: latex
%%% TeX-master: t
%%% End: 
