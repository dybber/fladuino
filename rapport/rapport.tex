\documentclass[a4paper,oneside, draft]{memoir}
\usepackage[T1]{fontenc}
\usepackage[utf8]{inputenc}
\usepackage[english]{babel}

% bedre orddeling Gør at der som minimum skal blive to tegn på linien ved
% orddeling og minimum flyttes to tegn ned på næste linie. Desværre er værdien
% anvendt af babel »12«, hvilket kan give orddelingen »h-vor«.
\renewcommand{\englishhyphenmins}{22} 

\usepackage{colortbl}  % Bruges til at farve celler, rækker mv. i tabeller
\usepackage{pdflscape} % Gør landscape-environmentet tilgængeligt
\usepackage{fixme}     % Indsæt "fixme" noter i drafts.
\usepackage{hyperref}  % Indsæter links (interne og eksterne) i PDF


\renewcommand{\ttdefault}{pcr} % Bedre typewriter font
%\usepackage[sc]{mathpazo}     % Palatino font
\renewcommand{\rmdefault}{ugm} % Garamond
\usepackage[garamond]{mathdesign}

%\overfullrule=5pt
%\setsecnumdepth{part}
\setcounter{secnumdepth}{-1} % Sæt overskriftsnummereringsdybde. Disable = -1.

\newcommand{\EDSL}{EDSL (Embedded Domain Specific Language) \renewcommand{\EDSL}{ EDSL }}

\title{Controlling embedded devices with funktional reactive programming}

\author{Martin Dybdal (dybber@dybber.dk), \\
Jesper Reenberg (jesper.reenberg@gmail.com) \\ og
Troels Henriksen (athas@sigkill.dk)}

\date{\today}
\pagestyle{plain}



\begin{document}
\maketitle

\begin{abstract}
  Robots are conventionally programmed in low-level imperative
  languages with no concepts like events, synchronicity or any of the
  advantages found in functional programming languages (like pattern
  matching). Reactive programming languages embedded in Haskell, like
  Frob \cite{frob99} and Yampa \cite{arrowsrobotsfrp02}, has been
  suggested for robot programming, but they require a complete Haskell
  runtime system, which is to large to fit on more
  resource-constrained devices. We tie these two ends together using,
  creating a highly declarative language for robot programming using a
  staged-compilation strategy like the one found in Flask
  \cite{flask08}, that makes it possible to run programs on devices
  with few resources. In the creation we've used the Flask-codebase as
  a starting point and huge parts of the code is left unchanged.
\end{abstract}


\section{Disposition}
\begin{itemize}
\item Abstract

\item Preface
  \begin{itemize}
  \item Brief intro to robot programming
    \begin{itemize}
     \item Sensors and Actuators
    \end{itemize}
  \item Brief Arduino introduction
  \item Brief Flask introduction
    \begin{itemize}
     \item The staged compilation strategy
    \end{itemize}
  \end{itemize}

\item Previous Work
  \begin{itemize}
  \item Frob
  \item Esterel
    (http://www.softwaresafety.net/Esterel.org/esterel.html)
  \item Lustre \cite{lustre91}
  \item Flask
  \end{itemize}

\item "`Our system"'
  \begin{itemize}
  \item Staged compilation
  \item Dataflows
  \item Node representation
  \item Devices
  \item Interrupts
  \item Event-queue
  \end{itemize}

\item Example programs

\item Conclusions

\item Bibliography

\item Appendixes
  \begin{itemize}
  \item A Flask tour
  \item Small guide to Arduino and electronics
  \end{itemize}
\end{itemize}  

\bibliographystyle{plain}
\bibliography{rapport}

\end{document}
