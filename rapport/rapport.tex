\documentclass[a4paper,oneside, draft]{memoir}
\usepackage[T1]{fontenc}
\usepackage[utf8]{inputenc}
\usepackage[english]{babel}

% bedre orddeling Gør at der som minimum skal blive to tegn på linien ved
% orddeling og minimum flyttes to tegn ned på næste linie. Desværre er værdien
% anvendt af babel »12«, hvilket kan give orddelingen »h-vor«.
\renewcommand{\englishhyphenmins}{22} 

\usepackage{colortbl}  % Bruges til at farve celler, rækker mv. i tabeller
\usepackage{pdflscape} % Gør landscape-environmentet tilgængeligt
\usepackage{fixme}     % Indsæt "fixme" noter i drafts.
\usepackage{hyperref}  % Indsæter links (interne og eksterne) i PDF


\renewcommand{\ttdefault}{pcr} % Bedre typewriter font
%\usepackage[sc]{mathpazo}     % Palatino font
\renewcommand{\rmdefault}{ugm} % Garamond
\usepackage[garamond]{mathdesign}

%\overfullrule=5pt
%\setsecnumdepth{part}
\setcounter{secnumdepth}{-1} % Sæt overskriftsnummereringsdybde. Disable = -1.

\newcommand{\EDSL}{EDSL (Embedded Domain Specific Language) \renewcommand{\EDSL}{ EDSL }}

\title{Controlling embedded devices with funktional reactive programming}

\author{Martin Dybdal (dybber@dybber.dk), \\
Jesper Reenberg (jesper.reenberg@gmail.com) \\ og
Troels Henriksen (athas@sigkill.dk)}

\date{\today}
\pagestyle{plain}



\begin{document}
\maketitle

\begin{abstract}
  Robots are conventionally programmed in low-level imperative
  languages with no concepts like events, synchronicity or any of the
  advantages found in functional programming languages (like pattern
  matching). Reactive programming languages embedded in Haskell, like
  Frob \cite{frob99} and Yampa \cite{arrowsrobotsfrp02}, has been
  suggested for robot programming, but they require a complete Haskell
  runtime system, which is to large to fit on more
  resource-constrained devices. We tie these two ends together using,
  creating a highly declarative language for robot programming using a
  staged-compilation strategy like the one found in Flask
  \cite{flask08}, that makes it possible to run programs on devices
  with few resources. In the creation we've used the Flask-codebase as
  a starting point and huge parts of the code is left unchanged.
\end{abstract}


\section{Disposition}
\begin{itemize}
\item Abstract

\item Preface
  \begin{itemize}
  \item Brief intro to robot programming
    \begin{itemize}
     \item Sensors and Actuators
    \end{itemize}
  \item Brief Arduino introduction
  \item Brief Flask introduction
    \begin{itemize}
     \item The staged compilation strategy
    \end{itemize}
  \end{itemize}

\item Previous Work
  \begin{itemize}
  \item Frob
  \item Esterel
    (http://www.softwaresafety.net/Esterel.org/esterel.html)
  \item Lustre \cite{lustre91}
  \item Flask
  \end{itemize}

\item "`Our system"'
  \begin{itemize}
  \item Staged compilation
  \item Dataflows
  \item Node representation
  \item Devices
  \item Interrupts
  \item Event-queue
  \end{itemize}

\item Example programs

\item Conclusions

\item Bibliography

\item Appendixes
  \begin{itemize}
  \item A Flask tour
  \item Small guide to Arduino and electronics
  \end{itemize}
\end{itemize}  

\bibliographystyle{plain}
\bibliography{rapport}

\appendix

\chapter{Small guide to Arduino and electronics}
Arduino is an Open Source project ... hardware ... software
... prototyping.

\fixme{<<Image of our Duemilanove>>}

There are different kinds of Arduino boards. All are based on
different ATmega* processors from Atmel. The older boards used serial
communication while the newer are connected to a computer through a
USB cable, but still using the same serial communication protocol
(i.e. serial through USB). The USB cable also provides power for the
board.  

We will describe the Arduino Duemilanove board we have used in the
sections that follow, but most (if not all) information will apply to
the other Arduino boards as well.

\section{Pins}
The Arduino board can connect to several external components through
diverse input and output pins. The pins are split in three sections on
the board \textit{power}, \textit{analog in} and \textit{digital}.

\subsection{Power pins}
\begin{itemize}
\item Gnd: Ground
\item 5v: constant 5v output
\item Reset: resets the board when connected to ground. \fixme{tror jeg i
    hvert fald..}
\item 3V3: ???
\item Vin: ???
\item Sixth pin: ???
\end{itemize}

\subsection{Analog input}
There are 6 analog input pins. It's possible to read a voltage level
from each of them using the function
\texttt{int analogRead(int pinnumber)} where \textit{pinnumber} is a
number between 0 and 5 indicating which pin to read from. The result is given
as an integer in the range from 0 to 1023.

\subsection{Digital pins}
There are 14 digital input and output pins.
\fixme{description of how to use them and special features of the
  different pins}


\section{Timers}
The ATmega168/ATmega328 processors has three different hardware
timers. One of them is used by arduino software to provide a
\texttt{millis()} function that gives us the amount of milliseconds
since the timer started and a \texttt{delay} function that actively
holds the processor occupied for a given time period.

\fixme{...which are 8 bit and 16 bit timers. Interference problems with the
PWM output pins.}

\section{Electronics}
\fixme{Maybe a little refresher on electronics (e.g. how each
  component are drawn in a diagram, if we are going to use diagrams in
  the article). }

\end{document}
