\documentclass[a4paper,oneside, draft]{memoir}
\usepackage[T1]{fontenc}
\usepackage[utf8]{inputenc}
\usepackage[danish]{babel}

% bedre orddeling Gør at der som minimum skal blive to tegn på linien ved
% orddeling og minimum flyttes to tegn ned på næste linie. Desværre er værdien
% anvendt af babel »12«, hvilket kan give orddelingen »h-vor«.
\renewcommand{\danishhyphenmins}{22} 

\usepackage{colortbl}  % Bruges til at farve celler, rækker mv. i tabeller
\usepackage{pdflscape} % Gør landscape-environmentet tilgængeligt
\usepackage{fixme}     % Indsæt "fixme" noter i drafts.
\usepackage{hyperref}  % Indsæter links (interne og eksterne) i PDF


\renewcommand{\ttdefault}{pcr} % Bedre typewriter font
%\usepackage[sc]{mathpazo}     % Palatino font
\renewcommand{\rmdefault}{ugm} % Garamond
\usepackage[garamond]{mathdesign}

%\overfullrule=5pt
%\setsecnumdepth{part}
\setcounter{secnumdepth}{-1} % Sæt overskriftsnummereringsdybde. Disable = -1.

\newcommand{\EDSL}{Embedded Domain Specific Language (herefter forkortet
  EDSL) \renewcommand{\EDSL}{ EDSL }}


\title{Synopsis}

\author{Martin Dybdal (dybber@dybber.dk), \\
Jesper Reenberg (jesper.reenberg@gmail.com) \\ og
Troels Henriksen (athas@sigkill.dk)}

\date{\today}
\pagestyle{plain}



\begin{document}
\maketitle

\section{Projekttitel}
"`Oversættelse af et reaktivt programmeringssprog til styring af
indlejrede kontrolsystemer."'

\section{Problemformulering}
Hvordan kan man oversætte et reaktivt \EDSL skrevet i Haskell, så det
kan køre på indlejrede systemer med begrænsede beregningsressourcer?

\section{Uddybning}

I det følgende vil problemformuleringen blive uddybet i et motivations- og
implementationsafsnit.

\subsubsection{Motivation}
Programmering af indlejrede systemer til brug i kontrolsystemer og robotter sker
primært i en imperativ stil, hvor man via kommandoer aktivt skal lytte på
sensorer for at få fat i målinger. I programmeringssproget Frob\cite{frob99} er
det vist hvordan man i en reaktiv programmeringsstil kan udtrykke
styringsprogrammer der kommer tættere på specifikationsteksten. Dette sker ved
at man i stedet angiver en række \textit{hændelser} som systemet skal være
opmærksom på og hvilke \textit{reaktioner} systemet skal tage ved
disse. Fordelen ved dette, udover dem vi får ved at bruge et funktionelt
programmeringssprog, er at der er kortere vej fra specifikation til en
fungerende løsning.

På de fleste indlejrede systemer er det dog ikke muligt at køre Frob, da det
kræver en hel Haskell runtime (i hvert fald i den Frob-implementation der er
skrevet til \cite{frob99}) hvilket ikke er muligt med microcontrollernes
begrænsede kapacitet.  F.eks. har de fleste Arduino-boards \cite{arduino} en
16MHz Atmel ATmega168 microprocessor, med 14 KB flash hukommelse (16KB - 2KB
bootloader) og 512B EEPROM\footnote{Electrically Erasable Programmable Read-Only
  Memory}. Hvilket er langt fra nok til at rumme et almindeligt Haskell
runtimesystem.

I et andet domæne, sensornetværk, er det med projektet Flask\cite{flask08} vist
hvordan man kan oversætte, et \EDSL skrevet i Haskell, ned til kode der
kan køre på sådanne enheder med begrænset hukommelses- og
regnekapacitet. 

I vores projekt vil vi kombinere disse to ideer, så vi kan få et reaktivt
programmeringssprog i samme stil som Frob oversat, til at kunne køre på
Arduino-boards. Dette vil vi gøre ved at tage udgangspunkt i Flask.  


\subsubsection{Implementation}
Da vi har valgt Arduino-boards som målplatform (vil vores målsprog
som vi oversætter til) være "`Arduino Programming Language"', som er et
simpelt imperativt sprog med C++/Java lignende syntaks.

Vi vil starte med at tilpasse Flask, så det oversætter vores \EDSL til
Arduino-sproget, og lave tilpasninger så det passer til domænet. Det
er dog ikke sikkert at dette er en god strategi, så hvis det bliver for
besværligt vil vi i stedet starte helt fra bunden med vores egen
implementation.

I sensornetværk hvor Flask anvendes, sker der primært dataopsamling og
databehandling. I kontrolsystemer er der også brug for reaktioner på
det opsamlede data. Udfordringen bliver altså at ændre Flask så man
også kan beskrive reaktioner.

\section{Læringsmål}

Projektet skal gøre de studerende i stand til at:

\begin{enumerate}

\item Forklare hvad FRP er, samt forklare fordele og ulemper ved at bruge FRP.
  
q\item Forstå hvordan Flask fungerer og vise forståelsen ved at tilpasse Flask
 q til kontrolstyringsdomænet og Arduino-platformen.
  
\item Udvikle et større, struktureret, fleksibelt og veldokumenteret system i
  Haskell.
  
\item Foretage en systematisk evaluering af et sådant \EDSL.
  
\end{enumerate}


\section{Afgrænsninger}
\begin{enumerate}

\item Vi forventer ikke at programmer skrevet i vores programmeringssprog skal
  overholde realtime--krav. Dette betyder at effektiviteten af implementationen
  ikke er en væsentlig del af opgaven og at vores \EDSL ikke vil have primitiver
  til at angive realtime-krav.

\item Vi har ikke tænkt os at skrive noget større program i sproget. Vi har i
  stedet tænkt os at vise udtryksmulighederne, via mindre
  eksempelprogrammer. (At skrive et større program vil dog være en mulig
  udvidelse af projektet, hvis vi får tid)

\item Vi vil ikke lave nogen kvalitativ undersøgelse af hvor "`godt"' og
  "`anvendeligt"' vores sprog er til domænet.

\end{enumerate}

\section{Arbejdsopgaver}
Projektet løber fra uge 6 til og med uge 24.

\begin{enumerate}

\item Forudgående undersøgelser (1-2 uger).
  
  \begin{itemize}

  \item Find litteratur og læs om Flask og Arduino.

  \item Lav små programmer i Arduino-programmeringssproget, for at lære hvordan
    det bruges, så vi ved hvilken type kode der skal genereres.

  \item Undersøge allerede eksisterende robotprogrammeringssprog og almindelige
    arbejdsopgaver indenfor robotprogrammering.

  \end{itemize}

\item Definition af vores EDSL (hvor meget skal/kan tages fra Frob?)(3
  uger).  \label{item:opgaver:definer-syntaks}
  \begin{itemize}

  \item Definer syntaks og semantik

  \end{itemize}

\item Tilpasning af Flask (7 uger).
  \begin{itemize}

  \item Lav parser til det EDSL vi har defineret.

  \item Ret kodegeneratoren, så der genereres kode Arduino-sproget i
    stedet for nesC.

  \end{itemize}


\item Lav eksempler på hvordan sproget bruges i praksis (3 uger).
  \label{item:opgaver:lav-eksempler}

  \begin{itemize}

  \item Skriv selv programmer i Arduino-sproget eller find eksempler

  \item Sammenlign eksempelprogrammerne med tilsvarende skrevet direkte til
    Arduino-programmingssproget.  

  \end{itemize}
    
\item Færdiggøre rapport (2 uger).
  
\item Afpudsning af rapport (3 dage).
  
\item Buffer og tid til at arbejde på mulige udvidelser af projektet (uge 20-24).

\end{enumerate}


\section{Mulige udvidelser af projektet ("`nice to have"')}

\begin{enumerate}

\item Lav en Arduino-simulator, f.eks. ved at lave en Arduino $\rightarrow$
  Player/Stage oversætter.

\item Udvidelse af sproget til at skrive programmer der kører på flere robotter
  v.h.a. intern kommunikation med wifi/bluetooth/andet (flocking).

\item Skriv et større eksempelprogram i vores programmeringssprog.

\item Lav en formel semantik for sproget.

\end{enumerate}


% \begin{landscape}
% \begin{center}

%   \definecolor{uofsgreen}{rgb}{.125,.5,.25}
%   \definecolor{natvidgreen}{rgb}{.196,.364,.239}
%   \definecolor{lightgrey}{rgb}{.6,.6,.6}
%   \definecolor{grey}{rgb}{.4,.4,.4}


% \begin{tabular}{ccccccccccccccccccc}
% \multicolumn{19}{c}{Uge nr.} \\
% 6 & 7 & 8 & 9 & 10 & 11 & 12 & 13 & 14 & 15 & 16 & 17 & 18 &
% 19 & 20 & 21 & 22 & 23 & 24\\ \hline
% \multicolumn{3}{c}{\cellcolor{black} } & \multicolumn{8}{l}{Synopsis} \\  
% & & \multicolumn{2}{c}{\cellcolor{red}} & \multicolumn{8}{l}{Haskell-bindings} \\  
%  & & & \multicolumn{3}{c}{\cellcolor{lightgrey}} &

%  \multicolumn{8}{l}{Rapport: Indledning} \\
%  & & & & & \multicolumn{6}{c}{\cellcolor{grey}} &
%  \multicolumn{8}{l}{Rapport: Analyse-del} \\
% & & & & & & \multicolumn{6}{c}{\cellcolor{blue}} &
%  \multicolumn{7}{l}{Implementation} \\
%  & & & & & & &  \multicolumn{6}{c}{\cellcolor{magenta}} &
%  \multicolumn{6}{l}{Definition af syntaks} \\
% \multicolumn{10}{r}{Udvikling af eksempelprogrammer} & \multicolumn{5}{c}{\cellcolor{cyan}}
%  \\
% \multicolumn{15}{r}{Afpudsning af rapport} & \cellcolor{yellow}
%  \\
% \multicolumn{16}{r}{Buffer-uger} & \multicolumn{3}{c}{\cellcolor{green}} \\  
% \end{tabular}
% \end{center}
% \end{landscape}

\appendix 

\begin{landscape}
\chapter{Tidsplan}
\begin{center}

  \definecolor{uofsgreen}{rgb}{.125,.5,.25}
  \definecolor{natvidgreen}{rgb}{.196,.364,.239}
  \definecolor{lightgrey}{rgb}{.6,.6,.6}
  \definecolor{grey}{rgb}{.4,.4,.4}


\begin{tabular}{ccccccccccccccccccc}
\multicolumn{19}{c}{Uge nr.} \\
6 & 7 & 8 & 9 & 10 & 11 & 12 & 13 & 14 & 15 & 16 & 17 & 18 &
19 & 20 & 21 & 22 & 23 & 24\\ \hline \hline
\multicolumn{3}{c}{\cellcolor{black} \color{white}  Synopsis } \\  
& & \multicolumn{2}{c}{\cellcolor{cyan} \color{white}} &
\multicolumn{6}{l}{Lær Flask og Arduino at kende} \\
& & & \multicolumn{3}{c}{\cellcolor{lightgrey} \color{white} } &
\multicolumn{6}{l}{Rapport: Indledning} \\
& & & & &  \multicolumn{4}{c}{\cellcolor{magenta} \color{white} Definition af
  syntaks} \\
& & & & & \multicolumn{9}{c}{\cellcolor{grey} \color{white} Rapport: Analyse-del} \\

& & & & & & \multicolumn{7}{c}{\cellcolor{blue} \color{white} Implementation} \\
\multicolumn{10}{r}{} & \multicolumn{5}{c}{\cellcolor{red} Udvikling
  af eksempler}
 \\
\multicolumn{14}{r}{Rapport: Afpudsning} & \multicolumn{2}{c}{\cellcolor{yellow}}
 \\
\multicolumn{16}{r}{} & \multicolumn{3}{c}{\cellcolor{green} Buffer-uger} \\  
\end{tabular}
\end{center}
\end{landscape}

\bibliographystyle{plain}
\bibliography{synopsis}

\end{document}
