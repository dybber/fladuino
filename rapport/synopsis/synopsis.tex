\documentclass[a4paper,oneside, draft]{memoir}
\usepackage[T1]{fontenc}
\usepackage[utf8]{inputenc}
\usepackage[danish]{babel}

\usepackage{colortbl}  % Bruges til at farve celler, rækker mv. i tabeller
\usepackage{pdflscape} % Gør landscape-environmentet tilgængeligt
\usepackage{fixme}     % Indsæt "fixme" noter i drafts.


\renewcommand{\ttdefault}{pcr} % Bedre typewriter font
%\usepackage[sc]{mathpazo}     % Palatino font
\renewcommand{\rmdefault}{ugm} % Garamond
\usepackage[garamond]{mathdesign}

%\overfullrule=5pt
%\setsecnumdepth{part}
\setcounter{secnumdepth}{-1} % Sæt overskriftsnummereringsdybde. Disable = -1.


\title{Synopsis}

\author{Martin Dybdal (dybber@dybber.dk), \\
Jesper Reenberg (jesper.reenberg@gmail.com) \\ og
Troels Henriksen (athas@sigkill.dk)}

\date{\today}
\pagestyle{plain}



\begin{document}
\maketitle

\section{Projekttitel}
"`Oversættelse af et reaktivt programmeringssprog til styring af
indlejrede kontrolsystemer."'

\section{Problemformulering}
Hvordan kan man oversætte et reaktivt DSL indlejret i Haskell, så det
kan køre på indlejrede systemer med begrænsede beregningsresurser?
\fixme{brugen af ordet "`indlejret"' på to forskellige måder er
  irriterende, kan vi finde et andet ord?}

\section{Uddybning}

I det følgende vil problemformuleringen blive uddybet i et motivations- og
implementationsafsnit.

\subsection{Motivation}
Programmering af indlejrede systemer til brug i kontrolsystemer og
robotter sker primært i en imperativ stil, hvor man via kommandoer
aktivt skal lytte på sensorer for at få fat i målinger. I
programmeringssproget Frob\cite{frob99} er det vist hvordan man i en reaktiv
programmeringsstil kan udtrykke styringsprogrammer der kommer tættere
på specifikationsteksten. Dette sker ved at man i stedet angiver en
række \textit{hændelser} som systemet skal være opmærksomme på og
systemets \textit{reaktioner} på disse hændelser. Fordelen ved dette
er, udover fordelene som vi får ved at bruge et funktionelt
programmeringssprog, at der er kortere vej fra specifikation til et
fungerende produkt.

På de fleste indlejrede systemer er det dog ikke muligt at køre Frob,
da det kræver en hel Haskell runtime (i hvert fald i den
Frob-implementation der er skrevet til \cite{frob99}) hvilket ikke er muligt med
microcontrollernes begrænsede kapacitet.  F.eks. har de fleste
Arduino-boards en 8-bit ATmel ATmega168 microprocessor, med 8 KB
programmerbar hukommelse. Hvilket er langt fra nok til at rumme et
almindeligt Haskell runtimesystem.

I et andet domæne, sensornetværk, er det med projektet Flask\cite{flask08} vist
hvordan man kan oversætte et DSL indlejret i Haskell ned til kode der
kan køre på sådanne enheder med begrænset hukommelses- og
regnekapacitet.

I vores projekt vil kombinere disse to ideer så vi kan få et reaktivt
programmeringssprog i samme stil som Frob oversat så det kan køre på
Arduino-boards.


\subsection{Implementation}
Da vi har valgt Arduino-boards som målplatform, vil vores målsprog
som vi oversætter til, være "`Arduino Programming Language"', som er et
simpelt imperativt sprog med C/Java lignende syntaks.

Vi vil starte med at tilpasse Flask, så det oversætter vores EDSL til
Arduino-sproget og lave tilpasninger, så det passer til domænet. Det
er dog ikke sikkert dette er en god strategi, så hvis dette bliver for
besværligt vil vi i stedet starte helt fra bunden med vores egen
implementation.

\section{Læringsmål}

Projektet skal gøre de studerende i stand til at:

\begin{enumerate}
\item Forklare hvad FRP er, samt forklare fordele og ulemper ved at bruge FRP.

\item Forstå hvordan Flask fungerer og vise forståelsen ved at
  tilpasse Flask til kontrolstyringsdomænet og Arduino-platformen.

\item Udvikle et større struktureret, fleksibelt og veldokumenteret system i Haskell.

\item Foretage en systematisk evaluering af et sådant DSEL. \fixme{"et
    sådant DSEL"', hvilket?}
\end{enumerate}


\section{Afgrænsninger}
\begin{enumerate}
\item Vi forventer ikke at programmer skrevet i vores
  programmeringssprog skal overholde realtime--krav. Dette betyder at
  effektiviteten af implementationen ikke er en væsentlig del af
  opgaven og at vores EDSL ikke vil have primitiver til at angive
  realtime-krav.
\item Vi har ikke tænkt os at skrive noget større program i vores sprog. Vi har
  i stedet tænkt at vise udtryksmulighederne, via mindre eksempelprogrammer. (At
  skrive et større program vil dog være en mulig udvidelse af projektet, hvis vi
  får tid)
\item Vi vil ikke lave nogen kvalitativ undersøgelse af hvor "`godt"'
  og "`anvendeligt"' vores sprog er til domænet.
\end{enumerate}

\section{Arbejdsopgaver}

\fixme{Revider tidsmålene så de passer ind med 3 mennesker. Definer også faste
  datoer og visualiser hvordan de overlapper hinanden.}
Projektet løber fra uge 6 til og med uge 24.

\begin{enumerate}

\item Undersøge allerede eksisterende robotprogrammeringssprog og almindelige
  arbejdsopgaver indenfor robotprogrammering.

\item Forudgående undersøgelser (1 uge).
  
  \begin{itemize}
    
  \item Find ud af om vi kan få lov at låne Scorpion-robotterne og hvor
    vidt Player/Stage driveren til dem er funktionel nok til det er sjovt.
    
  \item Alternativt: skaf en anden robot (Roomba?)
    
  \end{itemize}

\item Lave Haskell-bindings til Player/Stage (FFI til C-biblioteket libplayerc)
  (1 uge).
  \label{item:opgaver:ffi-bindings}

\item Definere syntaks og semantik for vores EDSL (3 uger).
  \label{item:opgaver:definer-syntaks}

\item Implementere det definerede EDSL vha. de i \ref{item:opgaver:ffi-bindings}
  implementerede Haskell-bindings til Player/Stage og den i
  \ref{item:opgaver:definer-syntaks} definerede syntaks og semantik (2 uger).

\item Lav eksempler på hvordan sproget bruges i praksis. (2 uger)
  \label{item:opgaver:lav-eksempler}

  \begin{enumerate}
  
  \item Sammenlign eksempelprogrammerne med tilsvarende skrevet direkte til
    Player/Stage i C, C++ eller Python Robotics.
  
  \end{enumerate}
  
  
\item Brug de i \ref{item:opgaver:lav-eksempler} lavede eksempler til at teste
  om implementationen opfylder den i \ref{item:opgaver:definer-syntaks}
  definerede syntaks (1 uge).
  
\item Færdiggøre rapport.(2 uger)
  
\item Afpudsning af rapport. (3 dage)
  
\item Buffer og tid til at arbejde på mulige udvidelser af projektet. (uge 20-24)
\end{enumerate}


\section{Mulige udvidelser af projektet ("`nice to have"')}
\begin{enumerate}

\item Udvidelse af sproget til at skrive programmer til flere robotter (flocking).

\item Skriv et større eksempelprogram i vores programmeringssprog.

\item Lav en formel semantik for sproget

\end{enumerate}

\bibliographystyle{plain}
\bibliography{synopsis}


% \begin{landscape}
% \begin{center}

%   \definecolor{uofsgreen}{rgb}{.125,.5,.25}
%   \definecolor{natvidgreen}{rgb}{.196,.364,.239}
%   \definecolor{lightgrey}{rgb}{.6,.6,.6}
%   \definecolor{grey}{rgb}{.4,.4,.4}


% \begin{tabular}{ccccccccccccccccccc}
% \multicolumn{19}{c}{Uge nr.} \\
% 6 & 7 & 8 & 9 & 10 & 11 & 12 & 13 & 14 & 15 & 16 & 17 & 18 &
% 19 & 20 & 21 & 22 & 23 & 24\\ \hline
% \multicolumn{3}{c}{\cellcolor{black} } & \multicolumn{8}{l}{Synopsis} \\  
% & & \multicolumn{2}{c}{\cellcolor{red}} & \multicolumn{8}{l}{Haskell-bindings} \\  
%  & & & \multicolumn{3}{c}{\cellcolor{lightgrey}} &

%  \multicolumn{8}{l}{Rapport: Indledning} \\
%  & & & & & \multicolumn{6}{c}{\cellcolor{grey}} &
%  \multicolumn{8}{l}{Rapport: Analyse-del} \\
% & & & & & & \multicolumn{6}{c}{\cellcolor{blue}} &
%  \multicolumn{7}{l}{Implementation} \\
%  & & & & & & &  \multicolumn{6}{c}{\cellcolor{magenta}} &
%  \multicolumn{6}{l}{Definition af syntaks} \\
% \multicolumn{10}{r}{Udvikling af eksempelprogrammer} & \multicolumn{5}{c}{\cellcolor{cyan}}
%  \\
% \multicolumn{15}{r}{Afpudsning af rapport} & \cellcolor{yellow}
%  \\
% \multicolumn{16}{r}{Buffer-uger} & \multicolumn{3}{c}{\cellcolor{green}} \\  
% \end{tabular}
% \end{center}
% \end{landscape}


\begin{landscape}
\begin{center}

  \definecolor{uofsgreen}{rgb}{.125,.5,.25}
  \definecolor{natvidgreen}{rgb}{.196,.364,.239}
  \definecolor{lightgrey}{rgb}{.6,.6,.6}
  \definecolor{grey}{rgb}{.4,.4,.4}


\begin{tabular}{ccccccccccccccccccc}
\multicolumn{19}{c}{Uge nr.} \\
6 & 7 & 8 & 9 & 10 & 11 & 12 & 13 & 14 & 15 & 16 & 17 & 18 &
19 & 20 & 21 & 22 & 23 & 24\\ \hline \hline
\multicolumn{3}{c}{\cellcolor{black} \color{white}  Synopsis } \\  
& & \multicolumn{2}{c}{\cellcolor{cyan} \color{white}} & \multicolumn{6}{l}{Haskell-bindings} \\
& & & \multicolumn{3}{c}{\cellcolor{lightgrey} \color{white} } & \multicolumn{6}{l}{Rapport: Indledning} \\
& & & & & \multicolumn{6}{c}{\cellcolor{grey} \color{white} Rapport: Analyse-del} \\
& & & & & &  \multicolumn{6}{c}{\cellcolor{magenta} \color{white} Definition af
  syntaks} \\
& & & & & & & \multicolumn{6}{c}{\cellcolor{blue} \color{white} Implementation} \\
\multicolumn{10}{r}{} & \multicolumn{5}{c}{\cellcolor{red} Udvikling
  af eksempler}
 \\
\multicolumn{14}{r}{Rapport: Afpudsning} & \multicolumn{2}{c}{\cellcolor{yellow}}
 \\
\multicolumn{16}{r}{} & \multicolumn{3}{c}{\cellcolor{green} Buffer-uger} \\  
\end{tabular}
\end{center}
\end{landscape}



\end{document}
