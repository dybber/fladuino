\documentclass[10pt,a4paper,final,oneside,openany,article,twocolumn]{memoir}

%BASIC PACKAGES
\usepackage[english]{babel} % last language decides document language!
\usepackage[utf8x]{inputenc}          %text encoding
\usepackage{amsmath,amssymb, amsbsy}  % math
\usepackage{graphicx}
\usepackage[british]{isodate}
\usepackage[style=alphabetic,natbib=true]{biblatex}
\usepackage[sc]{mathpazo}
\usepackage{microtype}        % AWESOME typography!
\usepackage[draft]{fixme}


\bibliography{../bibliography/rapport}

\newsubfloat{figure}

%FONT
\usepackage[T1]{fontenc}
\usepackage{palatino}              % font : garamond
\linespread{1.05}                  % Palatino needs more leading (space between lines)
\usepackage[scaled]{beramono}

%PDF OUTPUT
\usepackage{hyperref}             % clickable url's in PDF-output
\hypersetup{
    linkcolor=black,
    citecolor=black,
    filecolor=black,
    urlcolor=black
}

%DOCUMENT INFO
\title{\vspace{-1.5cm}
  Reactive programming of embedded devices 
}
\author{
  Martin Dybdal -- \texttt{dybber@dybber.dk}\\
  Troels Henriksen -- \texttt{athas@sigkill.dk} \\
  Jesper Reenberg -- \texttt{reenberg@diku.dk}
}

%\datebritish
\date{26th October 2010}

\begin{document}
\twocolumn[\maketitle
\begin{onecolabstract}
  \vspace{-0.2cm} Embedded control systems are conventionally
  programmed in low-level imperative languages, with no concepts like
  events or synchronicity, as provided by functional reactive
  languages, such as Frob or Yampa. These languages though depends on
  a complete runtime system, which does not fit on resource
  constrained embedded devices.

  For the domain of \textit{sensor-networks}, Geoffrey Mainland
  suggests a compilation-strategy which allows programs to be
  specified with Haskell, but without needing Haskells runtime
  system. The tasks performed on a sensor-network node is limited as
  there are no need of actuation. We extend on Mainlands work, to
  allow programs to listen for events and react on those events
  through actuators.
  
  We present \textit{Fladuino} (combination of Flask and Arduino)
  through a number of examples in the control-system domain.
  \vspace{1.5cm}
\end{onecolabstract}
]

\chapter{Introduction}
\cite{arrowsrobotsfrp02}



\chapter{Evaluation}

\chapter{Future Work}

\chapter{Conclusion}

\defbibheading{bibliography}{\chapter{Bibliography}} 
\printbibliography

\end{document}

