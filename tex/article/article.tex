\documentclass[10pt,a4paper,final,oneside,openany,article,twocolumn]{memoir}

%BASIC PACKAGES
\usepackage[english]{babel} % last language decides document language!
\usepackage[utf8x]{inputenc}          %text encoding
\usepackage{amsmath,amssymb, amsbsy}  % math
\usepackage{graphicx}
\usepackage[british]{isodate}
\usepackage[style=alphabetic,natbib=true]{biblatex}
\usepackage[sc]{mathpazo}
\usepackage{microtype}        % AWESOME typography!
\usepackage[draft]{fixme}


\bibliography{../bibliography/rapport}

\newsubfloat{figure}

%FONT
\usepackage[T1]{fontenc}
\usepackage{palatino}              % font : garamond
\linespread{1.05}                  % Palatino needs more leading (space between lines)
\usepackage[scaled]{beramono}

%PDF OUTPUT
\usepackage{hyperref}             % clickable url's in PDF-output
\hypersetup{
    linkcolor=black,
    citecolor=black,
    filecolor=black,
    urlcolor=black
}

%DOCUMENT INFO
\title{\vspace{-1.5cm}
  Reactive programming of embedded devices 
}
\author{
  Martin Dybdal -- \texttt{dybber@dybber.dk}\\
  Troels Henriksen -- \texttt{athas@sigkill.dk} \\
  Jesper Reenberg -- \texttt{reenberg@diku.dk}
}

%\datebritish
\date{26th October 2010}

\begin{document}
\twocolumn[\maketitle
\begin{onecolabstract}
  % Evt. reference til Frob, Yampa o.l.
  \vspace{-0.2cm} Embedded devices are conventionally programmed in
  low-level imperative languages, with no concepts like events,
  syncronisity or data-flow abstractions. The work done on functional
  reactive programming at Yale by Hudak et. al., demonstrates the power of
  using languages based on these principles for programming control
  systems.

  The Yale research group doesn't tackle the problem of executing
  their code on the more resource constrained embedded devices. Their
  implementations depends on the Haskell runtime system which has too
  high resource demands.  For the domain of \textit{sensor-networks},
  Geoffrey Mainland has shown how Haskell can be used as a
  meta-language solely for generating the low-level programs. With
  this compilation strategy, the overhead of the Haskell runtime
  system is avoided.

  The tasks performed on a sensor-network node is limited as there are
  no need of actuation. We extend on Mainlands work, to allow programs
  to listen for events and react on those events through
  actuators. Part of contribution is also a way of defining device
  drivers for new devices that can generate events (such as the click
  of a simple hardware button)

  We present \textit{Fladuino} (combination of Flask and Arduino)
  through a number of examples in the control-system domain.
  \vspace{1.5cm}
\end{onecolabstract}
]

% Keywords: functional reactive programming, meta-language,
%           object-language

\chapter{Introduction}
\cite{arrowsrobotsfrp02}



\chapter{Evaluation}

\chapter{Future Work}

\chapter{Conclusion}

\defbibheading{bibliography}{\chapter{Bibliography}} 
\printbibliography

\end{document}

